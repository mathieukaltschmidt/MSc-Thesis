\begin{center}

	\makeatletter
	\thispagestyle{plain}
	\LARGE\bfseries{\@title} \\
 	\vspace{4mm}
	\large\bfseries{\@author} \\
	\normalfont
	\vspace{4mm}
	\large{May 10th, 2021} \\
	\vspace{2mm}
	\large{Institute for Theoretical Physics \\
		University of Heidelberg } \\
	\makeatother
\end{center}

\normalsize
\noindent This short report summarizes the current state of the art of my Master thesis research carried out as a member of the Strongly Correlated Systems Group at the Institute for Theoretical Physics in Heidelberg  under the supervision of Prof. Jan M. Pawlowski. \\

\noindent  We compute non-perturbative quark spectral functions for QCD in the Landau gauge. The spectral densities $\rho_i$\ are obtained by iteratively solving the corresponding Dyson-Schwinger equation for the quark propagator for a given input gluon spectral function.\\
 %TODO: Describe features of the result in a few sentences
 For the computation of the contributing one-loop quark self energy $\Sigma(p^2)$\ we make use of spectral renormalization, a technique based on dimensional regularization which has been developed recently in our group  \cite{Horak2019, Wink2020, HorakPawlowskiWink2020}. First promising results for the Yang-Mills sector of QCD have successfully proven the power of this method \cite{HorakPapavassiliouPawlowskiWink2021}. With our complementary computations for the matter sector of QCD we aim at contributing to the long term goal of our project line: Obtaining a general understanding of the spectral properties of QCD (at finite temperature and density) and its corresponding  phase diagram.