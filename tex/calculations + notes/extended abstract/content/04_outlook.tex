\section{Outlook}
After finishing the computation in the classical vertex approximation it would be of course desirable to include the full vertex, as it naturally appears in the DSE . This of course complicates the computation, since another spectral representation, i.\,e. for the quark-gluon vertex, needs to be included in the calculation  of the one-loop digram and therefore another spectral integration needs to be performed for every iteration step.  \\ % TODO: Gibt es Ans�tze hierf�r -> Ask Jan
On longer terms, we want to extend our results to finite temperatures and densities. This can for example be examined within thermal field theory approaches such as the Matsubara formalism \cite{Matsubara1955}, where the momentum integration in the time  domain $p^0$ is replaced by a discrete sum over the respective Matsubara frequencies, giving rise to the usual quantum statistical distribution functions that directly depend on temperature and chemical potential. This is currently studied in a scalar theory setting in our group. \\
One possibility for improvements on the numerical side, especially if we wish for faster iteration times, would be to discretize the spectral integrands on a three-dimensional grid (in the $p/\omega-\lambda_A-\lambda$ parameter space) and interpolate the obtained data points using standard routines provided by Mathematica such as Hermite polynomials and B-splines.\\
Of course, the general aim of our project line is to have a precise knowledge of the spectral properties of all relevant QCD quantities. Once this is achieved, this  can be used to study and make predictions for various interesting QCD phenomena such as the dynamics of the quark-gluon plasma in relativistic heavy-ion collisions whose evolution is usually described in terms of hydrodynamical transport equations \cite{AyikNorenbergWolschin1977, Xu2019}.

