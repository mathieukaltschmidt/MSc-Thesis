%compiling with XeLaTeX
\documentclass[digital, %colors on
			   %print,  %for printing
			   %twoside, %turn off for digital version
			   openright, %chapter always start on the right page
			   parskip=half,
			   11pt]{mythesis}

\usepackage{bbold}


\begin{document}
\pagestyle{plain}

\section*{Correct computation of the res}
To account for the the splitting  into pole and continuous contributions to the spectral functions we need to compute the residue of $\Gamma^{(2)}$ in every iteration step since it enters the pole contribution as a prefactor, characterizing the associated contour integral enclosing the (isolated) pole.\\
We want to make use of the fact, that the residues of an inverse function can easily be computed, if we know the zeros of the original function, i.\,e.
\begin{equation}
	\text{If}\ f(a) = 0\ \text{is a first order root of}\ f \implies \operatorname{Res}_a \frac{1}{f} = \frac{1}{f'(a)},
\end{equation} 
for a complex function $f$ with $a\in\mathbb{C}$.
This translates in our case to
\begin{equation}
	 \eval{\det \left\{\Gamma_q^{(2)}(p^2)\right\}}_{p=p_{\mathrm{root}}} = 0 \implies \operatorname{Res}_{p_{\mathrm{root}}} G_q(p^2) = \eval{\frac{1}{\partial_{p^2}\Gamma_q^{(2)}(p^2)}}_{p=p_{\mathrm{root}}}.
\end{equation}
Note that we used the determinant here since we are working with a matrix-valued 2-point function, i.\,e.
\begin{equation}
\begin{aligned}
		\Gamma^{(2)}_q(p^2)  &= G_{q0}^{-1}(p^2) - \Sigma_q(p^2) \\
		&= i\slashed{p} A(p^2) + B(p^2),
\end{aligned}
\end{equation}
where $A$ and $B$\ are two scalar dressing functions needed to fully characterize the Dirac structure of the quark propagator. Its determinant can be computed as follows:
 \begin{equation}
\begin{aligned}
	 \det \left\{\Gamma_q^{(2)}(p^2)\right\} &= \det \left\{ipA(p^2) \begin{pmatrix}
\phantom{-}1 & \phantom{-}0 & \phantom{-}0 & \phantom{-}0\phantom{-}\\
\phantom{-}0 & \phantom{-}1 & \phantom{-}0 & \phantom{-}0\phantom{-}\\
\phantom{-}0 & \phantom{-}0 & -1 & \phantom{-}0\phantom{-}\\
\phantom{-}0 & \phantom{-}0 & \phantom{-}0 & -1\phantom{-}\\
\end{pmatrix} +  B(p^2)\begin{pmatrix}
\phantom{-}1 & \phantom{-}0 & \phantom{-}0 & \phantom{-}0\phantom{-}\\
\phantom{-}0 & \phantom{-}1 & \phantom{-}0 & \phantom{-}0\phantom{-}\\
\phantom{-}0 & \phantom{-}0 & \phantom{-}1 & \phantom{-}0\phantom{-}\\
\phantom{-}0 & \phantom{-}0 & \phantom{-}0 & \phantom{-}1\phantom{-}\\
\end{pmatrix} \right\} \\
	 &= \left(\left(ipA(p^2) + B(p^2\right)\left(-ipA(p^2) + B(p^2\right)\right)^2 \\
	 &= \left(p^2A^2(p^2) + B^2(p^2)\right)^2 \\
		&= p^4 A^4(p^2) + 2p^2B^2(p^2)A^2(p^2) + B^4(p^2)\\
		&\overset{!}{=} 0.
\end{aligned}
\end{equation}
Here we used the fact, that we do not have to consider the spatial momenta explicitly since their information can always be restored from Lorentz invariance, implying $\slashed{p} \rightarrow p^0\gamma_0$ with $p^0\equiv p$, the time-like Euclidean momentum.
\end{document}