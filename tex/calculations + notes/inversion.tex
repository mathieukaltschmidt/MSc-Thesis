%compiling with XeLaTeX
\documentclass[digital, %colors on
			   %print,  %for printing
			   %twoside, %turn off for digital version
			   openright, %chapter always start on the right page
			   parskip=half,
			   11pt]{mythesis}

\usepackage{bbold}


\begin{document}
\pagestyle{plain}

\section*{Inverting the propagator to find the right dependencies}
Schematically the quark propagator DSE reads 
\begin{equation}
\begin{aligned}
		\Gamma^{(2)}_q(p^2) = S^{-1}_q(p^2) &= S_{q0}^{-1}(p^2) - \Sigma_q(p^2) \\
		&= i\slashed{p} A(p^2) + B(p^2),
\end{aligned}
\end{equation}
where $A$ and $B$\ are two scalar dressing functions needed to fully characterize the Dirac structure of the quark propagator. 
For the computation of the respective spectral functions we need the inverse of this expression. This inversion works as follows:
\begin{equation}
\begin{aligned}
	 S_q(p^2) &= \frac{1}{i\slashed{p} A(p^2) + B(p^2)}  \\
		&= \frac{-i\slashed{p} A(p^2) + B(p^2)}{(i\slashed{p} A(p^2) + B(p^2))(-i\slashed{p} A(p^2) + B(p^2))} \\
		&= \frac{-i\slashed{p} A(p^2) + B(p^2)}{p^2A^2(p^2) + B^2(p^2)} \\
		&= -i\slashed{p} \mathcal{A}(p^2; A,B) + \mathcal{B}(p^2; A,B)
\end{aligned}
\end{equation}
with the explicit definitions:
\begin{equation}
	\begin{aligned}
		\mathcal{A}(p^2) &= \frac{A(p^2)}{p^2A^2(p^2) + B^2(p^2)} \\
		\mathcal{B}(p^2) &= \frac{B(p^2)}{p^2A^2(p^2) + B^2(p^2)}
	\end{aligned}
\end{equation}
\color{MScRed} In the literature it is common to write 
\begin{equation}
	S_q(p^2) =  \frac{1}{i\slashed{p} A(p^2) + B(p^2)} = \frac{Z(p^2)}{i\slashed{p} + M(p^2)}
\end{equation} 
with 
\begin{equation}
	 Z(p^2) = \frac{1}{A(p^2)}\qquad \text{and}\qquad M(p^2) = \frac{B(p^2)}{A(p^2)}
\end{equation}
We can benchmark our results by comparing with known results for $Z$ and $M$ from perturbation theory, which can be restored from our computations by considering the chiral limit $\lambda_i \rightarrow 0$.\\
 \normalcolor
From the dependencies derived above, we may understand how to extract the respective part of the spectral functions $\rho_1$ and $\rho_2$ by comparing the derived expression with the spectral representation of the quark propagator:
\begin{equation}
	S_q(p^2) = -i\slashed{p} \mathcal{A}(p^2; A,B) + \mathcal{B}(p^2; A,B) \equiv \slashed{p}\underbrace{\int\limits_0^\infty\frac{\dd\lambda}{\pi}\frac{\lambda\rho_1(\lambda)}{p^2+\lambda^2}}_{\equiv\ -i\mathcal{A}(p^2)} + \underbrace{\int\limits_0^\infty\frac{\dd\lambda}{\pi}\frac{\lambda\rho_2(\lambda)}{p^2+\lambda^2}}_{\equiv\ \mathcal{B}(p^2)}
\end{equation}
With this knowledge we might understand how to extract the respective spectral function as a function of $\mathcal{A}(p^2;A,B)$ and $\mathcal{B}(p^2;A,B)$. We already finished the computation of $A(p^2)$ and $B(p^2)$ and therefore also of $\mathcal{A}(p^2;A,B)$ and $\mathcal{B}(p^2;A,B)$.\\
In our numerical calculation, the spectral function is extracted from the imaginary part of the analytically continued propagator as follows:
\begin{equation}
	\rho(\omega) = 2 \operatorname{Im}\left[S_{q}\left(-i(\omega + i0^+); \mathcal{A}_{\mathrm{cont}}, \mathcal{B}_{\mathrm{cont}}\right)\right]
\end{equation}
This allows us to access the respective spectral functions by projecting on the parts of the propagator that are proportional to $-i\slashed{p}$ and unity.\\
In total this yields
\begin{align}
	\rho_1(\omega) &= 2 \operatorname{Im}\left[\Pi_{\slashed{p}} S_{q}\left(-i(\omega + i0^+); \mathcal{A}_{\mathrm{cont}}, \mathcal{B}_{\mathrm{cont}}\right)\right] = 2 \operatorname{Im}\left[-i\mathcal{A}\left(-i(\omega + i0^+)\right)\right], \\
	\rho_2(\omega) &= 2 \operatorname{Im}\left[\Pi_{\mathbb{1}} S_{q}\left(-i(\omega + i0^+); \mathcal{A}_{\mathrm{cont}}, \mathcal{B}_{\mathrm{cont}}\right)\right]  = 2 \operatorname{Im}\left[\mathcal{B}\left(-i(\omega + i0^+)\right)\right],
\end{align}
where the respective projectors are constructed such that they project on the respective parts of the propagator,
\begin{align}
	\Pi_{\slashed{p}}\left(\cdots\right) &= \operatorname{Tr}_{\mathrm{D}}\left[\frac{\slashed{p}}{p^2} \left(\cdots\right)\right],\\
	\Pi_{\mathbb{1}}\left(\cdots\right) &= \operatorname{Tr}_{\mathrm{D}}\left[\left(\cdots\right)\right].
\end{align}
Here the index $\mathrm{D}$\ refers to the Dirac trace and we used the fact that the trace over an odd amount of gamma matrices vanishes.
\newpage
\section*{Explicit derivation}
Perform explicitly the correct limit $\epsilon\rightarrow 0^+$ for the retarded propagator $S(-i(\omega + i0^+))$:
\begin{equation}
	\begin{aligned}
		\lim\limits_{\epsilon\rightarrow 0^+} \frac{1}{-(\omega + i\epsilon)^2 + \lambda^2} &= 	\lim\limits_{\epsilon\rightarrow 0^+} \frac{1}{-\omega^2 + \epsilon^2 + \lambda^2 - 2i\epsilon\omega} \\
		&= 	\lim\limits_{\epsilon\rightarrow 0^+} \frac{-\omega^2 + \epsilon^2 + \lambda^2 + 2i\epsilon\omega}{(\epsilon^2 + \lambda^2 - \omega^2)^2 + 4\omega^2\epsilon^2}
	\end{aligned}
\end{equation}
Now look explicitly at the imaginary part of this expression:
\begin{equation}
	\begin{aligned}
		\Im \lim\limits_{\epsilon\rightarrow 0^+} \frac{1}{-(\omega + i\epsilon)^2 + \lambda^2} &= \lim\limits_{\epsilon\rightarrow 0^+}  \frac{2\epsilon\omega}{(\epsilon^2 + \lambda^2 - \omega^2)^2 + 4\omega^2\epsilon^2} \\
		&= \frac{1}{2\omega}  \lim\limits_{\epsilon\rightarrow 0^+} \frac{\epsilon}{\left(\frac{\epsilon^2 + \lambda^2 - \omega^2}{2\omega}\right)^2 + \epsilon^2}\\
		&= \frac{\pi}{2\omega}\delta\left(\frac{\lambda^2-\omega^2}{2\omega}\right) \\
		&= \frac{\pi}{2\omega}\left(\eval{\frac{\partial\left(\frac{\lambda^2-\omega^2}{2\omega}\right)}{\partial\lambda}}_{\lambda=\omega}\right)^{-1}\delta\left(\lambda-\omega\right)\\
		&= \frac{\pi}{2\omega}\delta\left(\lambda-\omega\right)
	\end{aligned}
\end{equation}
Here we used the expression for the Fourier transform of the $\delta$-function and assumed $\omega > 0$ and $\lambda > 0$. In total this means that:
\begin{equation}
	\int\frac{\dd\lambda}{\pi}\frac{\lambda\rho(\lambda)}{p^2 + \lambda^2}\ \longrightarrow\ \int\frac{\dd\lambda}{\pi}\frac{\lambda\rho(\lambda)}{-\omega^2 + \lambda^2} + i\frac{\rho(\omega)}{2}
\end{equation}
This now trivially explains how we can extract the spectral function from the retarded propagator:
\begin{equation}
	\rho(\omega) =  2 \operatorname{Im}\left[S\left(-i(\omega + i0^+)\right)\right]
\end{equation}
\end{document}
