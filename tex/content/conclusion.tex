\chapter{Conclusions and Outlook}\label{chap:conclusion}
In this thesis, we investigated quark spectral functions and the analytical properties of the associated real-time correlation functions by iteratively solving the corresponding Dyson-Schwinger equation in a novel approach based on the manifestly gauge-invariant dimensional regularization scheme.\\
After introducing the required background knowledge on functional approaches to QFT in \chapref{chap:methods}, and on the non-perturbative sector of QCD in \chapref{chap:qcd}, we introduced the central functional relations in the context of this work, the Dyson-Schwinger equations for the gluon, ghost and the quark propagators, the latter one being the relevant equation to set up our computation. In \chapref{chap:results}, we presented the application of the spectral renormalization scheme at the concrete example of the quark self energy diagram, as the non-trivial one-loop contribution to the quark propagator DSE. By naive power counting, the diagram is superficially divergent, which left us with the task to employ an appropriate regularization scheme in order to guarantee finite integrands . After regularizing the usual loop-momentum integration via dimensional regularization and the application of a BPHZ-type subtraction scheme in order to modify the UV-behavior of the integrands, we were left with finite expressions for the spectral integrands $I_q^D(p)$ and $I_q^M(p)$. The final step before being able to solve the quark DSE, was to analytically continue the obtained expressions to real-time frequencies, since the spectral functions are obtained from the imaginary parts if the respective parts of the retarded quark propagator. Based on this setup, a procedure to iteratively solve the quark DSE by employing an initial ansatz for the quark spectral functions $\rho_q^D$ and $\rho_q^M$, that allows for a numerical treatment of the remaining spectral integrals, was put forward. For the input gluon spectral function we used an analytic result, that has been reconstructed from numerical data for the Euclidean quark propagator. In every iteration step, the quark DSE is solved and an updated result for the spectral functions was obtained, that served as input for the next iteration steps.\\
In the considered approximation, replacing the full 3-point vertex by the respective classical, bare counterpart and employing a quasi-classical ansatz for the quark spectral functions to initialize the iterative procedure, no convergent solution of the quark DSE could be obtained up to now. The development of the typical peak structure could indeed be observed for $n_{\mathrm{iter}}>2$, but the mandatory benchmark test, the comparison between the characteristic parts of the Euclidean quark propagator computed directly via the DSE and from the respective spectral functions, shows no exact agreement.
As we have seen, the current setup did not yet suffice to produce stable results. Additional work is therefore required to upgrade the numerical setup to ensure convergence of the iterative procedure. In addition, it would also be very important to examine the possibilities of choosing different initial conditions, first of course for the quarks, but also for the gluonic input, as done for example in  \cite{HorakPapavassiliouPawlowskiWink2021} for the spectral ghost DSE, where the impact of a whole range of slightly different gluon input spectral functions was studied. 
After finishing the computation in the classical vertex approximation it would be of course desirable to include the full vertex, as it naturally appears in the DSE . This of course complicates the computation, since another spectral representation, i.\,e. for the quark-gluon vertex, needs to be included in the calculation  of the one-loop digram and therefore another spectral integration needs to be performed for every iteration step.  \\ % TODO: Refer to refs
On longer terms, we want to extend our results to finite temperatures and densities. This can for example be examined within thermal field theory approaches such as the Matsubara formalism \cite{Matsubara1955}, where the momentum integration in the time  domain $p^0$ is replaced by a discrete sum over the respective Matsubara frequencies, giving rise to the usual quantum statistical distribution functions, that directly depend on temperature and chemical potential. This is currently studied in a scalar theory setting in our group. \\
One possibility for improvements on the numerical side, especially if we wish for faster iteration times, would be to discretize the spectral integrands on a three-dimensional grid (in the $p/\omega-\lambda_1-\lambda_2$ parameter space) and interpolate the obtained data points. In comparable studies this leads to faster iteration times of up to two orders of magnitude, cf. the appendix of \cite{HorakPapavassiliouPawlowskiWink2021}. Our current setup takes around four to seven hours to perform a single iteration.\\
Of course, the general aim of our project line is to have a precise knowledge of the spectral properties of all relevant QCD quantities. Once this is achieved, this  can be used to study and make predictions for various interesting QCD phenomena such as the dynamics of the quark-gluon plasma in relativistic heavy-ion collisions whose evolution is usually described in terms of hydrodynamical transport equations.

\cleardoublepage