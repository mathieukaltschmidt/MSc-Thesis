%{\hypersetup{allcolors=black}
\thispagestyle{plain}

\makeatletter

\begin{center}
\textbf{\small\@title} \\
\vspace{.1cm}
\@author \\
\end{center}

\makeatother

%Abstract in english language
\statement{\large Abstract}
In this thesis we compute non-perturbative quark spectral functions for quantum chromodynamics (QCD) in the Landau gauge. The spectral densities $\rho_i$\ are obtained by iteratively solving the corresponding Dyson-Schwinger equation (DSE) for the quark propagator for a given analytical input gluon spectral function that has been obtained from the reconstruction of numerical data for the Euclidean gluon propagator. \\ 
In the considered approximation, replacing the full 3-point vertex by the respective classical, bare counterpart and employing a quasi-classical ansatz for the quark spectral functions to initialize the iterative procedure, no convergent solution could be obtained up to now. The development of the typical peak structure could indeed be observed for $n_{\mathrm{iter}}>2$, but the mandatory benchmark test, the comparison between the characteristic parts of the Euclidean quark propagator computed directly via the DSE and from the respective spectral functions, shows no exact agreement.\\
 For the computation of the contributing one-loop quark self energy $\Sigma_q$\ we make use of spectral renormalization, a novel technique based on the gauge-invariant dimensional regularization scheme which has been developed recently in our group  \cite{Horak2019, Wink2020, HorakPawlowskiWink2020}. First promising results for the Yang-Mills sector of QCD have successfully proven the power of this method \cite{HorakPapavassiliouPawlowskiWink2021}. \\ With our complementary computations for the matter sector of QCD we aim at contributing to the long term goal of our project line: Obtaining a general understanding of the spectral properties of QCD (at finite temperature and density) and its corresponding  phase diagram.

\newpage
\thispagestyle{plain}

\makeatletter

\begin{center}
\textbf{\small Quark Spektralfunktionen aus Dyson-Schwinger Gleichungen mit spektraler Renormierung} \\
\vspace{.1cm}
\@author \\
\end{center}

\makeatother

%Abstract in german language
\statement{\large Zusammenfassung}
\begin{otherlanguage}{ngerman}
In dieser Arbeit berechnen
 wir nicht-st\"orungstheoretische Quark Spektralfunktionen f\"ur Quantenchromodynamik (QCD) in der Landau-Eichung. Die Spektralfunktionen $\rho_i$ werden bestimmt, indem wir die entsprechende Dyson-Schwinger Gleichung (DSE) f\"ur den Quark Propagator iterativ l\"osen und dabei f\"ur das Gluon einen analytischen Ausdruck der Spektralfunktion als Input verwenden, welcher mit Hilfe von Rekonstruktionstechniken aus numerischen Daten des euklidischen Gluon Propagators berechnet wurde. \\ %TODO: Ergebnisse diskutieren
In der untersuchten N\"aherung, in welcher der volle 3-Punkt Vertex durch sein klassisches Gegenst\"uck approximiert und ein quasi-klassischer Ansatz f\"ur die Quark Spektralfunktionen f\"ur die Initialisierung des iterativen Verfahrens gew\"ahlt wurde, konnte bisher keine konvergente L\"osung gefunden werden. Zwar kann das Ausbilden der typischen Peak-Struktur f\"ur $n_{\mathrm{iter}}>2$ beobachtet werden, der unbedingt notwendige Vergleich zwischen den beiden charakteristischen Anteilen des euklidischen Propagators, einmal direkt \"uber die DSE berechnet im Vergleich zu den entsprechenden Resultaten aus der Berechnung mittels der erhaltenen Spektralfunktionen, zeigt jedoch keine genaue \"Ubereinstimmung.   \\
F\"ur die Berechnung des Ein-Loop Beitrags $\Sigma_q$ zur Quark Propagator DSE verwenden wir spektrale Renormierung, eine neuartigen Methode basierend auf dem eichinvarianten Formalismus der dimensionalen Regularisierung, welche k\"urzlich in unserer Arbeitsgruppe entwickelt wurde, berechnet \cite{Horak2019, Wink2020, HorakPawlowskiWink2020}.
Erste vielversprechende Resultate f\"ur den Yang-Mills Sektor der QCD  zeigen die St\"arken dieser Methode auf \cite{HorakPapavassiliouPawlowskiWink2021}. \\Mit unseren Berechnungen f\"ur den Materie-Sektor der QCD wollen wir zum Langzeit-Ziel unserer Kollaboration beitragen, ein vollumf\"angliches Verst\"andnis der spektralen Eigenschaften der QCD (bei endlicher Temperatur und Dichte) sowie des zugeh\"origen Phasendiagramms zu erlangen.
\end{otherlanguage}
\vfill
\cleardoublepage
%}